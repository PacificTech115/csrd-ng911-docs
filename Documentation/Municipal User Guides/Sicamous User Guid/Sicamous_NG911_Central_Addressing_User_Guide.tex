\documentclass[11pt]{article}
\usepackage[margin=1in]{geometry}
\usepackage{setspace}
\usepackage{hyperref}
\usepackage{xcolor}
\usepackage{booktabs}
\usepackage{longtable}
\usepackage{tabularx}
\usepackage{array}
\usepackage{enumitem}
\usepackage{fancyhdr}
\usepackage{titlesec}
\usepackage{graphicx}
\usepackage{float}

\setstretch{1.15}

% Pacific Tech Systems brand palette
\definecolor{BrandNavy}{HTML}{10293A}
\definecolor{BrandTeal}{HTML}{0B8B9A}
\definecolor{BrandAqua}{HTML}{4FC3CF}
\definecolor{BrandSlate}{HTML}{3F5568}
\definecolor{BrandLight}{HTML}{EEF7F8}

\hypersetup{
  colorlinks=true,
  linkcolor=BrandTeal,
  urlcolor=BrandTeal,
  citecolor=BrandTeal
}

\pagestyle{fancy}
\fancyhf{}
\setlength{\headheight}{24pt}
\lhead{\textcolor{BrandSlate}{\textbf{NG911 Central Addressing System}}}
\rhead{\textcolor{BrandTeal}{\textbf{Sicamous User Guide}}}
\cfoot{\thepage}
\renewcommand{\headrulewidth}{0.8pt}
\renewcommand{\headrule}{\hbox to\headwidth{\color{BrandAqua}\leaders\hrule height \headrulewidth\hfill}}

\titleformat{\section}{\Large\bfseries\color{BrandNavy}}{\thesection}{0.75em}{}
\titleformat{\subsection}{\large\bfseries\color{BrandTeal}}{\thesubsection}{0.6em}{}
\titleformat{\subsubsection}{\normalsize\bfseries\color{BrandSlate}}{\thesubsubsection}{0.5em}{}

\setlist[itemize]{itemsep=0.3em,topsep=0.3em,leftmargin=1.6em}
\setlist[enumerate]{itemsep=0.4em,topsep=0.35em,leftmargin=1.8em}

\newcommand{\shotplaceholder}[2]{
\begin{center}
\fcolorbox{BrandAqua}{BrandLight}{
\parbox[c][6.0cm][c]{0.92\textwidth}{
\centering
\textbf{Screenshot Placeholder}\\[0.4em]
\textbf{#1}\\[0.5em]
\textit{Insert image: #2}
}}
\end{center}
}

\begin{document}
\hypersetup{pageanchor=false}
\begin{titlepage}
\thispagestyle{empty}
\begin{center}
\vspace*{1.2cm}
{\Huge\bfseries\color{BrandNavy} NG911 Central Addressing System}\\[0.35cm]
{\Large\bfseries\color{BrandTeal} Sicamous User Guide}\\[0.9cm]
{\large\color{BrandSlate} Author: John Soliman}\\[0.15cm]
{\normalsize\color{BrandSlate} Pacific Tech Systems}\\[0.15cm]
{\normalsize\color{BrandSlate} Version: 1.0}\\[1.1cm]
{\normalsize\color{BrandSlate} For: Sicamous NG911 Editors}\\[1.2cm]
\includegraphics[width=\textwidth]{branding.png}
\end{center}
\vfill
\noindent{\color{BrandTeal}\rule{\textwidth}{1.2pt}}\\[0.25cm]
\noindent{\color{BrandSlate}\small This guide provides step-by-step instructions for accessing and editing the Sicamous NG911 address layer through the CSRD portal and ArcGIS Pro.}
\end{titlepage}
\hypersetup{pageanchor=true}

\tableofcontents
\newpage

\section{Purpose and scope}
This guide explains the standard Sicamous workflows for editing NG911 central addressing data:

\textbf{Method 1: ArcGIS Pro}
\begin{enumerate}
\item Log in to the CSRD portal.
\item Open the Sicamous NG911 group.
\item Open the designated Sicamous editing feature layer item.
\item Choose \textbf{Open in ArcGIS Pro}.
\item Edit attributes/geometry and save edits.
\item Verify changes in the Feature Layers Data tab on the CSRD Portal.
\end{enumerate}

\textbf{Method 2: Experience Builder Web App}
\begin{enumerate}
\item Log in to the CSRD portal.
\item Open the Sicamous NG911 group.
\item Open the Web app Portal Item and click on View.
\item Edit and search records directly in the web browser.
\end{enumerate}
\textbf{Note:} The screenshots in this guide are representative examples of the portal/ArcGIS Pro workflows.

\section{System access details}
\subsection{Portal and item URLs}
\begin{longtable}{p{4.0cm}p{10.0cm}}
\toprule
\textbf{Name} & \textbf{URL} \\
\midrule\endhead
CSRD Portal Home & \href{https://apps.csrd.bc.ca/hub/home}{https://apps.csrd.bc.ca/hub/home} \\
Sicamous NG911 Group & \href{https://apps.csrd.bc.ca/hub/home/group.html?id=d86ec2e6c271497eaf8983af5ce577e1#overview}{Open Sicamous NG911 Group} \\
Editing Layer Item & \href{https://apps.csrd.bc.ca/hub/home/item.html?id=f820d1ba962846d1bd71fa0d3c975043}{Open Sicamous editing feature layer item} \\
Web App Item & \href{https://apps.csrd.bc.ca/hub/home/item.html?id=2a6307820c874d9bbfd7289a694408d4}{Open Sicamous Experience Builder Web App} \\
\bottomrule
\end{longtable}

\subsection{User credentials}
\begin{longtable}{p{3.2cm}p{4.0cm}p{5.8cm}}
\toprule
\textbf{Role} & \textbf{Username} & \textbf{Password} \\
\midrule\endhead
Viewer & \texttt{Sicamous} & \texttt{Sicamous\_2024} \\
Editor & \texttt{Sicamous\_Editing} & \texttt{SicamousEdit!@2026} \\
\bottomrule
\end{longtable}

\subsection{Role guidance}
\begin{itemize}
\item Use the \textbf{Viewer} account for read-only review and validation.
\item Use the \textbf{Editor} account for production edits.
\item Only edit in the designated Sicamous editing layer.
\end{itemize}

\section{Step-by-step editing workflow: Method 1 (ArcGIS Pro)}
\clearpage
\subsection{Step 1: Log in to the CSRD portal}
\begin{enumerate}
\item Open a browser and go to the CSRD Portal Home URL.
\item Click \textbf{Sign In}.
\item Enter your Sicamous credentials.
\item Confirm successful sign-in.
\end{enumerate}
\begin{figure}[H]
\centering
\setlength{\fboxsep}{0pt}
\setlength{\fboxrule}{0.5pt}
\fbox{\includegraphics[width=0.95\textwidth]{\detokenize{1-CSRD Portal Home Page.png}}}
\caption{CSRD Portal home page}
\end{figure}
\noindent\textit{\textbf{Annotation:} Click \textbf{Sign In} in the top-right corner, then use your Sicamous Viewer or Editor account.}

\clearpage
\subsection{Step 2: Navigate to the Sicamous NG911 group}
\begin{enumerate}
\item Open the Sicamous group URL.
\item Confirm group access and content visibility.
\item Locate the Sicamous editing feature layer item.
\end{enumerate}
\begin{figure}[H]
\centering
\setlength{\fboxsep}{0pt}
\setlength{\fboxrule}{0.5pt}
\fbox{\includegraphics[width=0.95\textwidth]{\detokenize{2-Portal Groups Sicamous.png}}}
\caption{Portal groups view}
\end{figure}
\noindent\textit{\textbf{Annotation:} From \textbf{Groups}, open the Sicamous group.}

\begin{figure}[H]
\centering
\setlength{\fboxsep}{0pt}
\setlength{\fboxrule}{0.5pt}
\fbox{\includegraphics[width=0.95\textwidth]{\detokenize{3-Sicamous Group.png}}}
\caption{Group overview page}
\end{figure}
\noindent\textit{\textbf{Annotation:} Confirm group details and open the group content list.}

\clearpage
\subsection{Step 3: Open the editing feature layer item}
\begin{enumerate}
\item Open the Sicamous editing feature layer item page.
\item Verify item details and ownership.
\item Confirm this is the intended editable layer.
\end{enumerate}
\begin{figure}[H]
\centering
\setlength{\fboxsep}{0pt}
\setlength{\fboxrule}{0.5pt}
\fbox{\includegraphics[width=0.95\textwidth]{\detokenize{4-Sicamous edit layer Overview.png}}}
\caption{Editing layer item details}
\end{figure}
\noindent\textit{\textbf{Annotation:} Verify item details before launching ArcGIS Pro.}

\clearpage
\subsection{Step 4: Open in ArcGIS Pro}
\begin{enumerate}
\item Click \textbf{Open in ArcGIS Pro} from the item page.
\item Allow ArcGIS Pro to launch.
\item Sign in to the CSRD Portal if prompted.
\item Confirm the Sicamous layer loads.
\end{enumerate}
\begin{figure}[H]
\centering
\setlength{\fboxsep}{0pt}
\setlength{\fboxrule}{0.5pt}
\fbox{\includegraphics[width=0.95\textwidth]{\detokenize{5-Confirm CSRD Portal Sign in in Arcgis Pro.png}}}
\caption{Confirm CSRD Portal sign in within ArcGIS Pro}
\end{figure}
\noindent\textit{\textbf{Annotation:} Use \textbf{Open in ArcGIS Pro} and authenticate if required.}

\clearpage
\subsection{Step 5: Edit records}
\begin{enumerate}
\item In ArcGIS Pro, open the \textbf{Edit} tab.
\item Select or create features as needed.
\item Update required fields.
\item Validate location and attributes.
\end{enumerate}
\begin{figure}[H]
\centering
\setlength{\fboxsep}{0pt}
\setlength{\fboxrule}{0.5pt}
\fbox{\includegraphics[width=0.95\textwidth]{\detokenize{6-Sicamous ArcGIS pro.png}}}
\caption{ArcGIS Pro layer loaded}
\end{figure}
\noindent\textit{\textbf{Annotation:} Confirm the layer is present in \textbf{Contents} and attribute table is accessible.}

\begin{figure}[H]
\centering
\setlength{\fboxsep}{0pt}
\setlength{\fboxrule}{0.5pt}
\fbox{\includegraphics[width=0.95\textwidth]{\detokenize{7-Sicamous ArcGISPro Edit.png}}}
\caption{ArcGIS Pro editing view}
\end{figure}
\noindent\textit{\textbf{Annotation:} Use \textbf{Modify Features} to edit geometry and attributes.}

\clearpage
\subsection{Step 6: Save edits}
\begin{enumerate}
\item Click \textbf{Save} on the ArcGIS Pro \textbf{Edit} tab.
\item Resolve any validation issues.
\item Confirm edits are committed.
\end{enumerate}
\begin{figure}[H]
\centering
\setlength{\fboxsep}{0pt}
\setlength{\fboxrule}{0.5pt}
\fbox{\includegraphics[width=0.95\textwidth]{\detokenize{8-Sicamous ArcGISPro Save Edits.png}}}
\caption{Save edits in ArcGIS Pro}
\end{figure}
\noindent\textit{\textbf{Annotation:} Finalize edits with \textbf{Save}.}

\clearpage
\subsection{Step 7: Verification of changes}
\begin{enumerate}
\item Return to the CSRD Portal in your web browser.
\item Navigate to the Sicamous editing feature layer item page.
\item Open the \textbf{Data} tab.
\item Review the records to confirm your recent edits from ArcGIS Pro are successfully reflected.
\end{enumerate}

\clearpage
\section{Step-by-step editing workflow: Method 2 (Experience Builder Web App)}
\textit{Note: Steps 1 and 2 (Logging in to the CSRD Portal and navigating to the Sicamous NG911 Group) are identical to Method 1.}

\subsection{Step 3: Open the Web App}
\begin{enumerate}
\item Open the Web App Portal Item from the group page.
\item Click on \textbf{View} to launch the application.
\end{enumerate}
\begin{figure}[H]
\centering
\setlength{\fboxsep}{0pt}
\setlength{\fboxrule}{0.5pt}
\fbox{\includegraphics[width=0.95\textwidth]{\detokenize{3A- Sicamous Group.png}}}
\caption{Web App Portal Item}
\end{figure}
\noindent\textit{\textbf{Annotation:} Launch the Experience Builder web application from its item details page.}

\clearpage
\subsection{Step 4: Create a new address}
\begin{enumerate}
\item In the Web App interface, click on the \textbf{Create} button.
\end{enumerate}
\begin{figure}[H]
\centering
\setlength{\fboxsep}{0pt}
\setlength{\fboxrule}{0.5pt}
\fbox{\includegraphics[width=0.95\textwidth]{\detokenize{4A- Sicamous Address Management Web App overview.png}}}
\caption{Web App Interface Overview}
\end{figure}

\begin{figure}[H]
\centering
\setlength{\fboxsep}{0pt}
\setlength{\fboxrule}{0.5pt}
\fbox{\includegraphics[width=0.95\textwidth]{\detokenize{4B- Sicamous Address Management Web App Home page Create.png}}}
\caption{Create Button}
\end{figure}
\noindent\textit{\textbf{Annotation:} Initiate the creation of a new address record.}

\clearpage
\subsection{Step 5: Place the address point}
\begin{enumerate}
\item On the right side of the page, click on the \textbf{Feature point Creation} button.
\item Click on the map to place the point exactly on the location of the new address.
\end{enumerate}
\begin{figure}[H]
\centering
\setlength{\fboxsep}{0pt}
\setlength{\fboxrule}{0.5pt}
\fbox{\includegraphics[width=0.95\textwidth]{\detokenize{5A- Sicamous Web App feature point.png}}}
\caption{Feature Point Creation}
\end{figure}
\noindent\textit{\textbf{Annotation:} Accurately drop the new address point on the map.}

\clearpage
\subsection{Step 6: Fill out the fields}
\begin{enumerate}
\item Fill out all required fields in the creation form.
\end{enumerate}
\textit{For more information on how to fill out the fields and which fields are automatically calculated and non-editable, refer to \hyperref[sec:appendix_edit_form]{Appendix: Web App Edit Form Guide}.}

\begin{figure}[H]
\centering
\setlength{\fboxsep}{0pt}
\setlength{\fboxrule}{0.5pt}
\fbox{\includegraphics[width=0.95\textwidth]{\detokenize{6A- Sicamous Web App Creation Form.png}}}
\caption{Creation Form}
\end{figure}
\noindent\textit{\textbf{Annotation:} Ensure all mandatory attribute data is entered for the new address.}

\clearpage
\subsection{Step 7: Click Create}
\begin{enumerate}
\item After reviewing the details, click \textbf{Create} to save the new address.
\end{enumerate}
\begin{figure}[H]
\centering
\setlength{\fboxsep}{0pt}
\setlength{\fboxrule}{0.5pt}
\fbox{\includegraphics[width=0.95\textwidth]{\detokenize{7A- Sicamous Web App Confirm Create.png}}}
\caption{Confirm Create}
\end{figure}
\noindent\textit{\textbf{Annotation:} Finalize the creation process.}

\clearpage
\subsection{Step 8: Return to Home page}
\begin{enumerate}
\item To edit an existing record instead, go back to the Home page by clicking the \textbf{Home} button.
\end{enumerate}
\begin{figure}[H]
\centering
\setlength{\fboxsep}{0pt}
\setlength{\fboxrule}{0.5pt}
\fbox{\includegraphics[width=0.95\textwidth]{\detokenize{8A- Sicamous Web AppHome Button.png}}}
\caption{Home Button}
\end{figure}
\noindent\textit{\textbf{Annotation:} Navigate back to the main interface to begin searching.}

\clearpage
\subsection{Step 9: Click on Edit Button}
\begin{enumerate}
\item Click on the \textbf{Edit} button to access the editing mode.
\end{enumerate}
\begin{figure}[H]
\centering
\setlength{\fboxsep}{0pt}
\setlength{\fboxrule}{0.5pt}
\fbox{\includegraphics[width=0.95\textwidth]{\detokenize{9A- Sicamous Web App Edit Button.png}}}
\caption{Edit Button}
\end{figure}
\noindent\textit{\textbf{Annotation:} Toggle the application into editing mode.}

\clearpage
\subsection{Step 10: Search for an address}
\begin{enumerate}
\item Click on the search bar on the map.
\item Search for the address by entering its Full Address or NGUID.
\end{enumerate}
\begin{figure}[H]
\centering
\setlength{\fboxsep}{0pt}
\setlength{\fboxrule}{0.5pt}
\fbox{\includegraphics[width=0.95\textwidth]{\detokenize{10A-WebApp Search Bar.png}}}
\caption{Search Bar}
\end{figure}
\noindent\textit{\textbf{Annotation:} Quickly locate existing records using the search widget.}

\clearpage
\subsection{Step 11: Click on the Search Result}
\begin{enumerate}
\item Click on the correct address from the search results to select it.
\end{enumerate}
\begin{figure}[H]
\centering
\setlength{\fboxsep}{0pt}
\setlength{\fboxrule}{0.5pt}
\fbox{\includegraphics[width=0.95\textwidth]{\detokenize{11A-WebApp Search Results.png}}}
\caption{Search Results}
\end{figure}
\noindent\textit{\textbf{Annotation:} Select the target address point to bring up its details.}

\clearpage
\subsection{Step 12: Edit the details}
\begin{enumerate}
\item Edit the details of the selected address on the edit form appearing on the right side of the page.
\end{enumerate}
\textit{For more information on how to fill out the fields and which fields are automatically calculated and non-editable, refer to \hyperref[sec:appendix_edit_form]{Appendix: Web App Edit Form Guide}.}

\begin{figure}[H]
\centering
\setlength{\fboxsep}{0pt}
\setlength{\fboxrule}{0.5pt}
\fbox{\includegraphics[width=0.95\textwidth]{\detokenize{12A-WebApp Edit Form.png}}}
\caption{Edit Form}
\end{figure}
\noindent\textit{\textbf{Annotation:} Update any incorrect attributes or attributes needing modification.}

\clearpage
\subsection{Step 13: Click Update}
\begin{enumerate}
\item Click \textbf{Update} to commit the changes to the database.
\end{enumerate}
\begin{figure}[H]
\centering
\setlength{\fboxsep}{0pt}
\setlength{\fboxrule}{0.5pt}
\fbox{\includegraphics[width=0.95\textwidth]{\detokenize{13A-WebApp Update Button.png}}}
\caption{Update Button}
\end{figure}
\noindent\textit{\textbf{Annotation:} Save your edits successfully.}

\clearpage
\appendix
\section{Appendix: Web App Edit Form Guide}
\label{sec:appendix_edit_form}

This section describes the \textbf{NG911\_Address\_Sicamous\_Edit} full form architecture within the Experience Builder Web App. The fields are grouped by section with user-friendly descriptions.

\textbf{Legend:}\\
\textbf{EDIT} = User enters/changes $\cdot$ \textbf{REQ} = Required $\cdot$ \textbf{RO} = Read-only $\cdot$ \textbf{CALC} = Arcade-calculated $\cdot$ \textbf{SERVER} = Server-calculated on create/update $\cdot$ \textbf{HIDDEN} = Not shown on the form

\subsection*{1) System Calculated Fields (No input needed) --- 9}
\textit{Section purpose: These are system outputs. They are calculated server-side and will populate/update after you create or update an address point. You do not type into these fields.}

\begin{itemize}
\item \textbf{Full Address} (RO + SERVER): The complete formatted address built from your address components (e.g., unit + civic + street).
\item \textbf{Agency} (RO + SERVER + CALC): The responsible agency/jurisdiction for this address record (e.g., Sicamous).
\item \textbf{NENA Globally Unique ID} (RO + SERVER): A unique identifier for the address record used for NG911/NENA tracking.
\item \textbf{Quality Check Status} (RO + SERVER): Indicates if the record passes or fails automated QA/QC checks.
\item \textbf{Date Updated} (RO + SERVER): Timestamp of the most recent edit saved to the address point.
\item \textbf{Additional Code} (RO + SERVER): A system-assigned code used for internal or NG911 reference (if applicable).
\item \textbf{Latitude} (RO + SERVER): Latitude derived from the point geometry.
\item \textbf{Longitude} (RO + SERVER): Longitude derived from the point geometry.
\item \textbf{Elevation} (RO + SERVER): Elevation value if the system calculates or stores it for the point.
\end{itemize}

\subsection*{2) Effective Dates --- 2}
\textit{Section purpose: Track when an address becomes valid and (optionally) when it is no longer valid.}

\begin{itemize}
\item \textbf{Effective Date} (EDIT): The date this address should be considered active/valid.
\item \textbf{Expiration Date} (EDIT): The date this address is no longer active (leave blank if still active).
\end{itemize}

\subsection*{3) Mandatory NENA Fields and Administrative Jurisdiction --- 7}
\textit{Section purpose: Required NG911 administrative areas. Most are auto-populated and locked to keep values consistent. (Calculated via Arcade in the webmap).}

\begin{itemize}
\item \textbf{Country} (RO + REQ + CALC): Country of the address (typically ``CA'').
\item \textbf{Province (A1)} (RO + REQ + CALC): Province/state administrative area (e.g., BC).
\item \textbf{Regional District (A2)} (RO + REQ + CALC): Regional district for the address location (e.g., Columbia Shuswap Regional District).
\item \textbf{Locality (A3)} (RO + REQ + CALC): Local municipality/locality (e.g., Sicamous).
\item \textbf{Discrepancy Agency ID} (RO + REQ + CALC): Identifier used to track agency responsibility and discrepancy workflows.
\item \textbf{Unincorporated Community (A4)} (EDIT): Used when the address is within an unincorporated area/community name.
\item \textbf{Neighborhood Community (A5)} (EDIT): Optional neighborhood name when there is a well-known defined neighborhood.
\end{itemize}

\subsection*{4) Main Address Components --- 12}
\textit{Section purpose: The core civic address pieces. These fields largely determine the Full Address output.}

\begin{itemize}
\item \textbf{Unit} (EDIT): Unit/suite/apartment identifier (e.g., ``12'', ``A'', ``Unit 3'').
\item \textbf{Address Number} (EDIT + REQ): Main civic number (e.g., ``123'').
\item \textbf{Address Number Suffix} (EDIT): Text after the civic number (e.g., ``A'' in ``123A'', ``1/2'' in ``123 1/2'').
\item \textbf{Street Name Pre Directional} (EDIT): Directional before street name (use full words: ``North'', ``Southwest'', etc.).
\item \textbf{Street Name Pre Modifier} (EDIT): Optional modifier before the name (e.g., ``Old'', ``Upper'', ``Lower'').
\item \textbf{Street Name Pre Type} (EDIT): Type before the name (rare) (e.g., ``Highway'' in ``Highway 1'').
\item \textbf{Street Name Pre Type Separator} (EDIT): Separator between pre-type and name (usually blank, space, or hyphen).
\item \textbf{Street Name} (EDIT + REQ): Core street name (e.g., ``Main'', ``Trans-Canada'', ``7'').
\item \textbf{Street Name Post Type} (EDIT): Street type after the name (use full words: ``Street'', ``Avenue'', ``Road'').
\item \textbf{Street Name Post Directional} (EDIT): Directional after the name/type (use full words: ``East'', ``Northwest'').
\end{itemize}

\subsection*{5) Legacy Street Components --- 4}
\textit{Section purpose: Store the legacy/previous street naming components for reference or historical comparison.}

\begin{itemize}
\item \textbf{Legacy Street Name Pre Directional} (EDIT): Legacy directional before the street name.
\item \textbf{Legacy Street Name} (EDIT): Legacy core street name.
\item \textbf{Legacy Street Name Type} (EDIT): Legacy street type (e.g., Rd, St, Ave as historically recorded).
\item \textbf{Legacy Street Name Post Directional} (EDIT): Legacy directional after the street name/type.
\end{itemize}

\subsection*{6) Internal System Reference and Notes --- 6}
\textit{Section purpose: Optional identifiers and notes that help link the address to other systems and add context for staff.}

\begin{itemize}
\item \textbf{Additional Data URI} (EDIT): A link/URI to external supporting information (documents, records, etc.).
\item \textbf{Feature ID} (EDIT): An internal ID for linking to another dataset/system.
\item \textbf{Parcel ID} (EDIT): Parcel identifier (PID/parcel number) associated with the address.
\item \textbf{Roll} (EDIT): Tax roll / assessment roll number if applicable.
\item \textbf{Address Notes} (EDIT): Free-text notes about the address (access, signage, special instructions, etc.).
\item \textbf{Alternate Access} (EDIT): Alternate access description (e.g., ``Access via rear lane'', ``Use service road'').
\end{itemize}

\subsection*{7) Unit -- Subproperty Components --- 5}
\textit{Section purpose: Extra sub-location info when ``Unit'' alone isn’t enough (multi-building sites, campuses, facilities).}

\begin{itemize}
\item \textbf{Building} (EDIT): Building identifier/name/number (e.g., ``Building B'').
\item \textbf{Floor} (EDIT): Floor level (e.g., ``3'', ``Mezzanine'').
\item \textbf{Room} (EDIT): Room identifier (e.g., ``201'', ``Control Room'').
\item \textbf{Seat} (EDIT): Seat/space identifier when relevant (e.g., arenas, venues).
\item \textbf{Additional Location Information} (EDIT): Any extra description to help locate the exact unit/subproperty.
\end{itemize}

\subsection*{8) Postal/MSAG/ESN --- 5}
\textit{Section purpose: Mailing and emergency service routing fields (where applicable/known).}

\begin{itemize}
\item \textbf{Postal Community Name} (EDIT): Community name used for mailing addresses.
\item \textbf{Postal Code} (EDIT): Postal code (e.g., ``V0E 2V0'').
\item \textbf{Postal Code Extension} (EDIT): Additional postal extension if used in a local standard.
\item \textbf{MSAG Community Name} (EDIT): Master Street Address Guide community name (legacy 911 routing reference).
\item \textbf{ESN} (EDIT): Emergency Service Number (legacy 911 field; used where still applicable).
\end{itemize}

\subsection*{9) Landmark/Placement --- 4}
\textit{Section purpose: Describe special address types and how/where the point was placed (useful for non-standard sites).}

\begin{itemize}
\item \textbf{Place Type} (EDIT): Type of place/feature (e.g., park, marina, campground, facility).
\item \textbf{Placement Method} (EDIT): How the point location was determined (e.g., rooftop, driveway, parcel centroid).
\item \textbf{Complete Landmark Name} (EDIT): Full landmark/facility name (e.g., ``Sicamous Community Hall'').
\item \textbf{Milepost} (EDIT): Milepost/km marker information for highway/rural addressing where applicable.
\end{itemize}

\clearpage
\section{Version control}
\begin{longtable}{p{2.0cm}p{2.8cm}p{3.8cm}p{5.0cm}}
\toprule
\textbf{Version} & \textbf{Date} & \textbf{Author} & \textbf{Change summary} \\
\midrule\endhead
1.0 & 2026-02-20 & John Soliman, Pacific Tech Systems & Initial Sicamous NG911 Central Addressing System user guide with representative annotated screenshots. \\
\bottomrule
\end{longtable}

\end{document}
