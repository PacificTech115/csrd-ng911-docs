\documentclass[11pt]{article}
\usepackage[margin=1in]{geometry}
\usepackage{setspace}
\usepackage{hyperref}
\usepackage{xcolor}
\usepackage{booktabs}
\usepackage{longtable}
\usepackage{tabularx}
\usepackage{array}
\usepackage{enumitem}
\usepackage{fancyhdr}
\usepackage{titlesec}
\usepackage{graphicx}
\usepackage{float}

\setstretch{1.15}

% Pacific Tech Systems brand palette
\definecolor{BrandNavy}{HTML}{10293A}
\definecolor{BrandTeal}{HTML}{0B8B9A}
\definecolor{BrandAqua}{HTML}{4FC3CF}
\definecolor{BrandSlate}{HTML}{3F5568}
\definecolor{BrandLight}{HTML}{EEF7F8}

\hypersetup{
  colorlinks=true,
  linkcolor=BrandTeal,
  urlcolor=BrandTeal,
  citecolor=BrandTeal
}

\pagestyle{fancy}
\fancyhf{}
\setlength{\headheight}{24pt}
\lhead{\textcolor{BrandSlate}{\textbf{NG911 Central Addressing System}}}
\rhead{\textcolor{BrandTeal}{\textbf{Salmon Arm User Guide}}}
\cfoot{\thepage}
\renewcommand{\headrulewidth}{0.8pt}
\renewcommand{\headrule}{\hbox to\headwidth{\color{BrandAqua}\leaders\hrule height \headrulewidth\hfill}}

\titleformat{\section}{\Large\bfseries\color{BrandNavy}}{\thesection}{0.75em}{}
\titleformat{\subsection}{\large\bfseries\color{BrandTeal}}{\thesubsection}{0.6em}{}
\titleformat{\subsubsection}{\normalsize\bfseries\color{BrandSlate}}{\thesubsubsection}{0.5em}{}

\setlist[itemize]{itemsep=0.3em,topsep=0.3em,leftmargin=1.6em}
\setlist[enumerate]{itemsep=0.4em,topsep=0.35em,leftmargin=1.8em}

\newcommand{\shotplaceholder}[2]{
\begin{center}
\fcolorbox{BrandAqua}{BrandLight}{
\parbox[c][6.0cm][c]{0.92\textwidth}{
\centering
\textbf{Screenshot Placeholder}\\[0.4em]
\textbf{#1}\\[0.5em]
\textit{Insert image: #2}
}}
\end{center}
}

\begin{document}
\hypersetup{pageanchor=false}
\begin{titlepage}
\thispagestyle{empty}
\begin{center}
\vspace*{1.2cm}
{\Huge\bfseries\color{BrandNavy} NG911 Central Addressing System}\\[0.35cm]
{\Large\bfseries\color{BrandTeal} Salmon Arm User Guide}\\[0.9cm]
{\large\color{BrandSlate} Author: John Soliman}\\[0.15cm]
{\normalsize\color{BrandSlate} Pacific Tech Systems}\\[0.15cm]
{\normalsize\color{BrandSlate} Version: 1.0}\\[1.1cm]
{\normalsize\color{BrandSlate} For: Salmon Arm NG911 Editors}\\[1.2cm]
\includegraphics[width=\textwidth]{branding.png}
\end{center}
\vfill
\noindent{\color{BrandTeal}\rule{\textwidth}{1.2pt}}\\[0.25cm]
\noindent{\color{BrandSlate}\small This guide provides step-by-step instructions for editing the Salmon Arm NG911 address layer through the CSRD portal via the Overwrite Hosted Feature Layer + ETL Reconcile workflow.}
\end{titlepage}
\hypersetup{pageanchor=true}

\tableofcontents
\newpage

\section{Purpose and scope}
This guide explains the standard Salmon Arm workflow for editing NG911 central addressing data:

\textbf{Editing Workflow: Overwrite Hosted Feature Layer + ETL Reconcile and Post to Central DB Feature Layer}
\begin{enumerate}
\item Go to CSRD Portal Home and log in with Editor Credentials.
\item Click on the content tab at the top of the page.
\item Click on SalmonArmOverwrite Hosted Feature layer.
\item On the Overview page of the hosted feature layer, click on update data.
\item Click on Overwrite Entire Feature Layer.
\item Upload zipped GDB file of updated Salmon Arm Database.
\item After uploading, click on Proceed.
\item Wait for the hosted feature layer to update.
\item View the Data tab of the hosted feature layer to verify if the updates were reflected.
\end{enumerate}

\section{System access details}
\subsection{Portal and item URLs}
\begin{longtable}{p{4.0cm}p{10.0cm}}
\toprule
\textbf{Name} & \textbf{URL} \\
\midrule\endhead
CSRD Portal Home & \href{https://apps.csrd.bc.ca/hub/home}{https://apps.csrd.bc.ca/hub/home} \\
Salmon Arm NG911 Group & \href{https://apps.csrd.bc.ca/hub/home/group.html?id=8bfe21794ce441a2af179df0abcda994#overview}{Open Salmon Arm NG911 Group} \\
Salmon Arm Central Database Feature layer & \href{https://apps.csrd.bc.ca/hub/home/item.html?id=aa1e950efc324d809ad4e6d005706a3e}{Open Salmon Arm Central Database Feature layer} \\
Salmon Arm Overwrite Hosted Feature Layer & \href{https://apps.csrd.bc.ca/hub/home/item.html?id=8cf1681215c540878625fdb4ec7434e4}{Open Salmon Arm Overwrite Hosted Feature Layer} \\
\bottomrule
\end{longtable}

\subsection{User credentials}
\begin{longtable}{p{3.2cm}p{4.0cm}p{5.8cm}}
\toprule
\textbf{Role} & \textbf{Username} & \textbf{Password} \\
\midrule\endhead
Viewer & \texttt{SalmonArm} & \texttt{SamlonArm@2024} \\
Editor & \texttt{Salmon\_Arm\_Editing} & \texttt{Salmonarm2026} \\
\bottomrule
\end{longtable}

\subsection{Role guidance}
\begin{itemize}
\item Use the \textbf{Viewer} account for read-only review and validation.
\item Use the \textbf{Editor} account for production edits and data overwrites.
\end{itemize}

\clearpage
\section{Step-by-step editing workflow}

\subsection{Step 1: Log in to the CSRD portal}
\begin{enumerate}
\item Go to the CSRD Portal Home.
\item Log in with your Editor Credentials.
\end{enumerate}
\begin{figure}[H]
\centering
\setlength{\fboxsep}{0pt}
\setlength{\fboxrule}{0.5pt}
\fbox{\includegraphics[width=0.95\textwidth]{\detokenize{1 portalhome.png}}}
\caption{CSRD Portal Home}
\end{figure}
\noindent\textit{\textbf{Annotation:} Authenticate with your assigned Editor account.}

\clearpage
\subsection{Step 2: Navigate to Content}
\begin{enumerate}
\item Click on the \textbf{Content} tab at the top of the page.
\end{enumerate}
\begin{figure}[H]
\centering
\setlength{\fboxsep}{0pt}
\setlength{\fboxrule}{0.5pt}
\fbox{\includegraphics[width=0.95\textwidth]{\detokenize{2 portal content tab.png}}}
\caption{Content Tab}
\end{figure}
\noindent\textit{\textbf{Annotation:} Access your organization's content listing.}

\clearpage
\subsection{Step 3: Open the Overwrite Hosted Feature Layer}
\begin{enumerate}
\item Locate and click on the \textbf{SalmonArmOverwrite} Hosted Feature layer.
\end{enumerate}
\begin{figure}[H]
\centering
\setlength{\fboxsep}{0pt}
\setlength{\fboxrule}{0.5pt}
\fbox{\includegraphics[width=0.95\textwidth]{\detokenize{3 overwrite layer item.png}}}
\caption{Hosted Feature Layer Item}
\end{figure}
\noindent\textit{\textbf{Annotation:} Open the item details page for the Salmon Arm Overwrite layer.}

\clearpage
\subsection{Step 4: Click Update Data}
\begin{enumerate}
\item On the Overview page of the hosted feature layer, click on \textbf{Update Data}.
\end{enumerate}
\textbf{Note:} It may take a few seconds for the Update Data button to appear.

\begin{figure}[H]
\centering
\setlength{\fboxsep}{0pt}
\setlength{\fboxrule}{0.5pt}
\fbox{\includegraphics[width=0.95\textwidth]{\detokenize{4 update data menu.png}}}
\caption{Update Data Menu}
\end{figure}
\noindent\textit{\textbf{Annotation:} Access the data update options.}

\clearpage
\subsection{Step 5: Overwrite Entire Feature Layer}
\begin{enumerate}
\item Click on \textbf{Overwrite Entire Feature Layer}.
\end{enumerate}
\begin{figure}[H]
\centering
\setlength{\fboxsep}{0pt}
\setlength{\fboxrule}{0.5pt}
\fbox{\includegraphics[width=0.95\textwidth]{\detokenize{5 Overwrite Option.png}}}
\caption{Overwrite Option}
\end{figure}
\noindent\textit{\textbf{Annotation:} Select the overwrite function.}

\clearpage
\subsection{Step 6: Upload the updated Database}
\begin{enumerate}
\item Upload the zipped GDB file of your updated Salmon Arm Database.
\end{enumerate}
\textbf{Important Note:} The Geodatabase name and the feature class name \textbf{must} match perfectly for the overwrite to succeed:
\begin{itemize}
\item \textbf{GDB Name:} \texttt{Default.gdb.zip}
\item \textbf{Feature Class Name:} \texttt{NG911\_AddressPoints\_SalmonArm\_Overwrite}
\end{itemize}
\begin{figure}[H]
\centering
\setlength{\fboxsep}{0pt}
\setlength{\fboxrule}{0.5pt}
\fbox{\includegraphics[width=0.95\textwidth]{\detokenize{6 upload gdb.png}}}
\caption{Upload GDB}
\end{figure}
\noindent\textit{\textbf{Annotation:} Drag and drop or browse to select your zipped GDB file.}

\clearpage
\subsection{Step 7: Proceed with Overwrite}
\begin{enumerate}
\item After the file finishes uploading, click on \textbf{Proceed}.
\end{enumerate}
\begin{figure}[H]
\centering
\setlength{\fboxsep}{0pt}
\setlength{\fboxrule}{0.5pt}
\fbox{\includegraphics[width=0.95\textwidth]{\detokenize{7 proceed overwrite.png}}}
\caption{Proceed Button}
\end{figure}
\noindent\textit{\textbf{Annotation:} Confirm the overwrite action.}

\clearpage
\subsection{Step 8: Wait for the update}
\begin{enumerate}
\item Wait for the hosted feature layer to finish updating. Do not close the window.
\end{enumerate}
\begin{figure}[H]
\centering
\setlength{\fboxsep}{0pt}
\setlength{\fboxrule}{0.5pt}
\fbox{\includegraphics[width=0.95\textwidth]{\detokenize{8 updating status.png}}}
\caption{Updating Status}
\end{figure}
\noindent\textit{\textbf{Annotation:} The system will process the uploaded GDB and replace the data.}

\clearpage
\subsection{Step 9: Verify the updates}
\begin{enumerate}
\item Once completed, go to the \textbf{Data} tab of the hosted feature layer.
\item Verify that the attributes and records correctly reflect your recent updates.
\item Done! The automated ETL will now safely reconcile and post this overwritten data to the Central Database Feature Layer.
\end{enumerate}
\begin{figure}[H]
\centering
\setlength{\fboxsep}{0pt}
\setlength{\fboxrule}{0.5pt}
\fbox{\includegraphics[width=0.95\textwidth]{\detokenize{9 data verification.png}}}
\caption{Data Tab Verification}
\end{figure}
\noindent\textit{\textbf{Annotation:} Confirm your data was overwritten successfully.}

\clearpage
\section{ETL Process Verification}
The automated ETL script runs on a scheduled basis every Saturday at 12:00 AM to synchronize your overwrites from the hosted feature layer directly to the Central Database Feature Layer.

After every scheduled run, the system automatically sends out a status email outlining the complete status of the ETL job, detailing all applied operations including Inserts, Updates, and Deletes. 

\begin{figure}[H]
\centering
\setlength{\fboxsep}{0pt}
\setlength{\fboxrule}{0.5pt}
\fbox{\includegraphics[width=0.95\textwidth]{\detokenize{10 ETL Process Verification Email.png}}}
\caption{ETL Process Verification Status Email}
\end{figure}
\noindent\textit{\textbf{Annotation:} Example of the automated status email detailing synchronization operations.}

\clearpage
\section{Version control}
\begin{longtable}{p{2.0cm}p{2.8cm}p{3.8cm}p{5.0cm}}
\toprule
\textbf{Version} & \textbf{Date} & \textbf{Author} & \textbf{Change summary} \\
\midrule\endhead
1.0 & 2026-02-22 & John Soliman, Pacific Tech Systems & Initial Salmon Arm NG911 Central Addressing System user guide outlining the Overwrite Hosted Feature Layer workflow. \\
\bottomrule
\end{longtable}

\end{document}
